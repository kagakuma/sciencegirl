\documentclass[a4paper,dvipdfmx,12pt]{jsarticle}
\title{猫の憂鬱}
\author{時岡結絃}
\date\today

\usepackage[dvipdfmx]{graphicx}
\usepackage[top=15truemm,bottom=15truemm,left=10truemm,right=10truemm]{geometry}
\usepackage{amsmath,amssymb}
\usepackage{color}


\begin{document}
\begin{titlepage}
\maketitle
\thispagestyle{empty}
\end{titlepage}


\section{登場人物}
恵美(めぐみ):一年生女子 主人公 片岡先輩を追いかけて同じ高校を受験し、同じ部活にいる 片岡先輩に片思いを続けている

村上先輩:三年生男子 大槻先輩にいつも絡まれているがまんざらでもない様子

大槻先輩:二年生女子 今回の話で様子がわかる 

片岡先輩:二年生男子 実験と講義が好きな変わり者 今日は不在

\section{猫の憂鬱} 
大槻先輩は実験をしない。紅茶を飲んで喋って帰るだけの人だ。でも片岡先輩も村上先輩も頭が上がらないらしい。

どこかミステリアスな雰囲気を醸し出している。長いストレートの黒髪に細いラインの華奢な身体と、無口ではないけれど慎重に話す姿が、紅茶に似合っている。コーヒーだと優雅さが喪われる気がする。

いつも実験を終えた先輩と一緒に、どこから買ってくるのか自分で持ってきたスコーンを片手に校舎の外を眺めている。そんなアンニュイな姿はとてもかっこいい。

校舎にはそこかしこで個人練に励む吹奏楽部の無秩序な音楽がけたたましく鳴り響いている。17時のゆったりした時間を完膚なきまでに潰してくれるようだ。そんな中でも、大槻先輩は自分を崩さない。

「理学部専用のノイズキャンセリング技術を確立しなくては」と冗談を飛ばすのが村上先輩の口癖だけど、あるときこんな風に静かに返したのを聞いた。

「そうかしら、鈴虫の鳴き声を不快だと感じる人に音楽はわからないと思うけれど」

\vspace{0.2in}

今日は片岡先輩がいないので、広い化学教室には私ひとりしかいない。実験をせずに、教科書とノートを前に少し背伸びをしていた。

化学にも必要になると片岡先輩から聞いて、量子力学に興味を持った。しかし、とりあえずニュートンあたりから攻めてみたのは失敗だったかもしれない。一般の人が興味を持つような「量子力学はこんなに不思議なんだ」という話ばかりが載っていて
実際に知りたいことは何も書いていなかった。

こうなればそろそろ専門書を読んでみるしかない、と決心した。先週から理学部の蔵書から黒い本を一冊拝借している。きちんと最初のページから数式を追って意味を理解しようと試みている。

当たり前だが、簡単ではない。とりあえずシュレディンガー方程式がいかにして浮かび上がり、簡単な場合の粒子の波動関数がどうなるかといったところまで読んだ。ふむふむと続きをわかったつもりになって16時30分を迎えた。

\vspace{0.2in}

大槻先輩はいつもきっかり16時30分に化学教室に来る。

「あら、今日はめぐみちゃんだけなのね」

「はい、片岡先輩、いないみたいで」

「そう。じゃあクッキーも二人ぶんでいいわね。紅茶、飲むでしょ?」

今日は村上先輩が文化祭の話し合いとかで会議に出ている。片岡先輩も他の部員も顧問も不在なので、珍しいことに私と大槻先輩だけだ。

\vspace{0.2in}

大槻先輩は隣にある準備室で紅茶を淹れに行った。茶葉は全て大槻先輩の私物だ。準備室においているものをその日の気分で必ず自分で抽出する。

もしかして英国貴族の末裔なのだろうか。この高校の学区は多くが住宅密集地で、貴族が住むような場所ではないのだけど。

こんな調子だから、話題も「今日は庭のバラが綺麗に咲きましたわ」とか「お父様の紹介でオペラのチケットが取れましたの」とかなのかと最初は思っていた。ところが実際の会話の内容は、非常に高度な物理の話ばかりである。

いつも村上先輩に一方的に絡んで静かに盛り上がっている。議論についていく村上先輩ももちろんすごい。でも論文を引いて毎日話題を持ってくる大槻先輩は怪物級の賢さだと確信する。

こんな調子なので、やっぱり話しかけづらい人なのかと思っていた。最初は自分から話しかけられなかったけれども、隣で紅茶を飲んでいると村上先輩との話を簡単に要約して教えてくれたりした。

それでもわからないことはもちろんある。恐る恐る質問をしてみると、嫌な顔もせず馬鹿にしたような態度も取らずにきちんと聞いて答えてくれる。

友達も多いらしく、昼休みの中庭に行ってみると大槻先輩を囲んで弁当を広げている女子の姿が見られる。

頭脳明晰、容姿端麗、性格柔和。欠点がない。いや、あるか。とっつきにくい人ではある。きっとモテるんだろうけど、誰かと付き合っていると言う話は一切聞かない。スキがない人だ。

\vspace{0.2in}

二人きりとなるとさすがに少し緊張する。いつものような難しい話を振られても理解できないことの方が多いし、バラやオペラにしたってさっぱりだ。いや、休日に観劇しているというのは私の妄想に過ぎないのだけれども。

「お手伝い、しましょうか?」

黒い本を閉じて化学準備室に顔を出してみる。棚に並んだ茶葉を選んでいる大槻先輩が見えた。

「そうね。じゃあ、ここのカップを出してポットのお湯で温めておいてくれる?」

先輩が手招きする。ここには先生の机もあるのだけど、今日は本当に一人もいない。

私はカップを、今度こそ落とさないように慎重に取り出した。この部屋には、先生がいつもカップラーメンを食べるために沸かしているポットがある。半分ほど残ったお湯を勝手に使わせてもらって、カップを温めた。

大槻先輩は、やかんに水道水を勢いよく入れてコンロにかけた。

「今日はめぐみちゃんだけだから、好きな茶葉を選んでいいよ」

そういって、店でも開けるのではないかというほど並んだ紅茶缶の前に私を立たせた。そう言われても、紅茶の種類なんて全くわからないのですけれども。

私は大槻先輩の紅茶トークを思い出しながら、知ったかぶりを決め込むことにした。

「そうですね。今日はミルクティって気分なので、ニルギリとかどうでしょうか」

そもそも毎日ミルクティって気分なので今日に限った話ではないのだが、私が選んでいいと言っているのだから誰も文句は言わないはず。

「そうね、いいチョイスだと思うわ。じゃあ今日はニルギリにしましょう。牛乳はまだ残っているはず」

とりあえず的外れなことは言っていないようで安心した。大槻先輩はミルクポットを片手に冷蔵庫を開け、やはりどこから買ってくるのか特濃ミルクを取り出してポットに注いだ。

いつだったか冷蔵庫を覗いたことがあるのだが、レモンやジャム、クロテッドクリームすらおいてあった。理学部はかなり好き勝手にやらせてもらっているが、紅茶とスコーンのために棚と冷蔵庫を占有するのはやりすぎなのではないだろうか。

大槻先輩はやかんの蓋を開け、じっとお湯の加減をみている。曰く

「茶葉が大事なのは当然。紅茶は水と温度だから、温度だけでもきちんと管理しないとね」

ということなので、並々ならぬこだわりを見せている。大槻先輩は美味しい紅茶のためには手を抜かない。でも不思議なことがある。英国貴族界ではメイドが紅茶を淹れるのでは?

ほどなくして大槻先輩が火を止め、四人ぶんの茶葉を入れたポットにお湯を注いだ。


時計より正確に化学教室へと立ち寄る大槻先輩だが、茶葉の抽出時間はさすがに砂時計で測るらしい。
ポットが化学教室に運ばれ、カバンからクッキーが出てきた。何回か砂時計が回転すると、温めておいたカップに紅茶が注がれる。
広い実験室に、大槻先輩と二人。先輩はいつも通りだろうけれど、私は気が気でない。緊張が高まってきた。何を話せばいいのかわからない。カップを持つ手が震えているのを悟られないように、必死で平静を保った。

\vspace{0.2in}

紅茶を一口いただくと、二人同時にカップをソーサーに置いた。ずっと何を話せばいいのかと考えていたが、それは取り越し苦労になった。

「今日は何の本を読んでいたの?」


「えっと、はい、量子力学の入門書を……」

そう言って恵美は黒い本を差し出した。

「ああーこの本ね。名著と言われているけど、そもそも量子力学は前提知識がないとあんまり楽しめないかもね」

最初の方をパラパラとめくって、こんなことも勉強したわねと言いながら先輩は語った。

「とりあえずどこまで読んだの?」

「まだ最初の方です。一次元箱ポテンシャルの場合について自分で解いて答え合わせはしてみました」

「そっか、すごいじゃない。高校一年生でここまでわかる人はそうそういないはずよ」

上品な笑みを浮かべながら本を恵美に返した。

「でもやっぱりシュレディンガー方程式が何でこの形になるのかって言うことについてはさっぱりわからなくて……」

「あーそれは私も疑問だったわ。みんなそうじゃないかしら」

「だった。ってことは、大槻先輩は知ってるんですか?」

「そうね、私なりの解釈というのは持っているつもり」

「そうなんですね。そこに到るまでは時間がかかりそうです」

クッキーを少しずつかじりながら、謙虚さを装って返答した。

沈黙の時間が流れる。実際にはセミの声と吹奏楽部の練習と野球部の掛け声が綯い交ぜになった混沌とした音楽が流れていたのだが、この教室だけ静まり返って時間が止まったかのように感じた。

何を話せばいいのだろうと恵美が戸惑っていると、大槻先輩がゆっくりとカップを置き、嬉しそうに言った。

「久しぶりに講義してみようかしら」

「講義、ですか?」

「そう。講義は片岡くんの領分だけど、今日くらいいいわよね」

「聞きたいです!」

今度は本心からそう答えた。

「じゃあ、これを飲み終わったらね」

大槻先輩はそう言って、二杯目の紅茶をカップに注ぎ、たっぷりと牛乳を入れた。

\vspace{0.2in}

大槻先輩がカーディガンを脱いで白衣を着ると、白墨を持って大きな黒板の前にたった。
「さて、どこから始めましょうか」


「そうですね。シュレディンガー方程式があれば解けるので、どうやって導出したのか、とかでしょうか」

「そうねーシュレディンガーの原著論文には、実は明確には書いてないのよ。ファインマンは「どこからも導けない。これはシュレディンガーの頭の中から出てくるのだ」って言っていたわけだし」


「そうなんですか。意外ですね。じゃあシュレディンガーの直観がたまたま合っていたって言うことなんですか?」

「そこまで当てずっぽうでもないわね。明確には書いていないけど、ちゃんと根拠はあるの。少しずつ説明するわね」

そう言うと、一本の数式を黒板に書き始めた。

\begin{equation}
E=\frac{p^2}{2m}+V
\end{equation}


「この式は見たことある?」

「いえ、でもすぐに導出できそうですね」

「そうね、pは運動量、Eは運動エネルギー、Vは位置エネルギーだと思ってもらえばいいかな。それぞれ、高校の物理学で習うような速度と質量の関係式に直せばすぐにこの式が正しいことはわかるんじゃないかな」

「そうですね、$E=1/2mv^2$で、$p=mv$だから……」


\begin{eqnarray}
p^2&=&m^2v^2 \\
E&=&\frac{1}{2} mv^2
\end{eqnarray}

$p^2$を$2m$で割ればこの式は成り立つことがすぐにわかる。


「この方程式は古典力学、つまり量子論以前の物理学で必ず成り立っていないといけない関係式。だから量子力学の世界でも、こう言う基本的な関係式は成り立っていてほしいと思うのは必然ね」

「必然、なんですか?」


「だってこれが成り立ってないと言うことはエネルギー保存則が成り立ってないってことじゃない?量子論の世界でエネルギー保存則が成り立ってないなら、巨視的に見たときにもズレが生じるはず」


必然、と言い切ってしまうところに若干の不安を覚えたが、聞いてみれば当たり前のことだった。

「でも、量子論の世界でエネルギー保存則が破れているかもしれない、って考えた人はいないわけじゃない。β崩壊の際にエネルギー保存則が成立していないということを「発見」して論文になったこともあるわけだし。結局、ニュートリノの質量が確定して解決したけどね」

いつも村上先輩に熱く語っているような調子で大槻先輩にもエンジンがかかったらしい。なるほど、という相槌を入れる間も無く次の話に進んでいく。

「シュレディンガーは、当時着目されていた波と粒子の二重性と言うものを敢えて無視したのではないかと考えられているの」

「無視、って、それが量子論の一番のポイントなんじゃないんですか?」

「そうね、でも実際にはそうじゃないことを見抜いていたのかもしれない。シュレディンガーは、量子の世界においては全てのものが波で表されると言う、ド・ブロイの思い切った仮説を受け入れたのね」

「粒子としての性質は無視したんですか?」

「とりあえず無視したと思ってもらっていいんじゃないかしら。でもそれが正解だったのね」

「やっぱり当てずっぽうじゃないですか」

「最初はそうかもね。でも結果的に正しければそれでいいのよ。そう言う世界なんだから」

そうなんだから仕方ないじゃない?と言うような笑顔を向けられて、納得せざるを得ない。実験が好きな恵美にとっては机上の空論のような感覚を覚えたのだが、そう言うモヤモヤは無視して話が次に飛んだ。

「で、全てのものは波で表されるんだから、まずその波動関数が与えられたと考えてみましょう」

\begin{equation}
\psi=\exp{2\pi i(x/\lambda-\nu t)}
\end{equation}

「これはわかるわよね?」

「波長$\lambda$、振動数$\nu$の波をあらわす関数ですね」

「そう。これに量子論の関係式を放り込んでみましょう。エネルギーと振動数の関係とド・ブロイの関係式かな。関係式をここに書いてみて?」

白墨をこちらに差し出すので、言われるままに書き出してみた

\begin{eqnarray}
E&=&h \nu \\
p&=&h/\lambda
\end{eqnarray}

「そうそう。上がアインシュタインの光電効果の式で、下がド・ブロイの関係式ね。じゃあこれをさっきの方程式に代入するとどうなるかしら」

講義と言いつつ、私の物理学演習のような時間になっている。代入するだけなので特に難しいことはない。すぐに次のような方程式を得た。

\begin{equation}
\psi=\exp{2\pi i(px/h-Et/h)}
\end{equation}

「じゃあ問題です。さっきの古典力学で成り立つ関係式、$E=p^2/2m$が成立するためには波動関数はどういう関係式を満たしていなければならないでしょうか?」

「関係式、ですか……」

ちょっと考えただけではすぐにはわからない。$p$も$E$も$\exp$の肩に乗っているので、もしかして$\log$を取ればいいのか?と思った。$\log\psi$と書こうとした私を見ていた大槻先輩が、一言忘れていたと言ってヒントをくれた。

「関係式じゃあ曖昧すぎたわね。この波動関数が、どのような微分方程式の解になっていないといけないか。それを考えてみましょう」

「なんかそれって鍵に合う錠前を作るみたいな話ですね」


「そうもいえるわね。でもとりあえずこれがゴール。もっと正確に言えば、偏微分方程式になっているはず」

偏微分方程式。ということは、波動関数を偏微分してみれば答えが見つかるかもしれない。

「$\psi$を$t$で一階偏微分してみますね」

指数関数の肩に乗っている係数を前に出すだけの作業だ。

\begin{equation}
\frac{\partial\psi}{\partial t} = \frac{-2\pi iE}{h} \psi
\end{equation}

「そうそう、次は$p$ね」

「ですね。$x$で二階偏微分してみます」

\begin{equation}
\frac{\partial^2\psi}{\partial x^2} = -\left(\frac{2\pi p}{h}\right)^2 \psi
\end{equation}

「$V$はどうするんですか?」

「それはもう少しあとで考えることにしましょう。それで、この二つの式を古典力学の関係式に当てはまるようにするにはどうすればいい?」

「そうですね、ちょっと整理します」

指数関数は微分しても関数の形が変わらないので、波動関数の中身はどうでもよくて係数だけいじればいいことに気づいた。まずは$E$と$p^2$の表示を考えてみる。

\begin{eqnarray}
E\psi&=&\frac{ih}{2\pi}\frac{\partial\psi}{\partial t} \\
\frac{p^2\psi}{2m}&=&-\left(\frac{h}{2\pi}\right)^2\frac{1}{2m}\frac{\partial^2\psi}{\partial x^2}
\end{eqnarray}

「その形まで来たならあと少しね。ここで$\psi$に$E$や$p^2$がかかっているように、位置エネルギーも$\psi$に掛けてしまいましょう」

そういうと綺麗な字で一つの式を追記した。

\begin{equation}
V\psi=V\psi
\end{equation}

「あと一息。この古典力学の関係式に戻してみましょう」


あとは係数を調整するだけで済みそうだ。私にもできる。えーっと$E=p^2/2m+V$だから

\begin{equation}
\frac{ih}{2\pi}\frac{\partial\psi}{\partial t} = -\left(\frac{h}{2\pi}\right)^2 \frac{1}{2m} \frac{\partial^2\psi}{\partial x^2}+ V\psi
\end{equation}


「これでゴールね。でもこれだとh/2πがたくさん出てきてみづらいので、ディラックの記号$\hbar=h/2\pi$を使うと

\begin{equation}
i\hbar \frac{\partial\psi}{\partial t} = -\frac{\hbar^2}{2m} \frac{\partial^2\psi}{\partial x^2}+ V\psi
\end{equation}

「ほら、できた」

「すごい、できた」

見たことのある方程式が目の前に現れた。自力で方程式を導出できた喜びが湧き上がってくる。

「おめでとう、これであなたも量子力学の世界に入ることができたわ」

にこやかに歓迎の挨拶を受けた。100年の時を超えて、先人の辿った道筋に再び踏み入れることができるのは感慨深い。

「シュレディンガーもこうやって導出したんでしょうか」

「そのことについては本人しか知り得ないことでしょうね。私はこうだと思っているけれども」

「それはそうですね、仮定が大胆ですし」

「波動しか考えないとか、古典力学で成り立つべき式に思い切って代入したとかね。でも他にももっと大胆な仮定があるの」
「どんなことですか?」

「例えば当時、アインシュタインの特殊相対性理論は十分に受け入れられていたと考えられているんだけど、この方程式を導くに当たって相対論的な考察は一切していないわよね。相対論はやった?」

「少しだけ、特殊の方だけは」

「じゃあ、少し宿題をあげる。シュレディンガー方程式は、確かに相対論的な考察をしていない。でも、この式自体はすでに相対論的になっているの」

「そうなんですか?ぱっと見じゃわかりませんね」

「そうね、だから宿題。それと、波動を仮定するなら当然考えるべきことは、きちんと干渉現象を説明できないといけないはずよね」
「そうですね。それに関しても何にも要請してない気がします。でも満たしているんですか?」

「満たしているわ。干渉って、要するに波の重ね合わせ、つまり方程式の線型性のことなのよ。シュレディンガー方程式に従う別の波動関数の定数倍を足し合わせても、シュレディンガー方程式は成り立つはず。これも宿題ね」

「そうですか。。じゃあ家で考えてきます。でもそれって大事なことなんですか?」

「そこなの。この方程式は、線形方程式になっている。つまり波の重ね合わせ、例えば当時知られていた干渉現象とか、そう言うことを説明するには線形でないといけない。でもそれを最初に要請しなくても、きちんと線形になっているのよ」

「不思議ですね……」

「シュレディンガーは当然深く考えたはずよね。でもきちんと様々な要請を満たす方程式を、シンプルかつ大胆な仮説を元に導き出せた。それがすごいことなの」

\vspace{0.2in}

一通り話し終わると、大槻先輩は白衣を脱いで白墨の粉を払い落とした。

「どうしてこの本を読んでるの?」

当然の質問だろう。素直に答えることにした。

「そうですね、私じゃまだ難しいとは思うのですが、分子軌道法について知りたいと思ったんです」

「やっぱり片岡くんね」

「いえ、そういう訳では……」

「片岡くん、かっこいいと思う?」


「かっこいい、かどうかですか?」

急に恋の話なんて、どうしたのだろう。大槻先輩が意地悪な目をしている。こんな表情もするんだ。もしかして、私の好きな気持ちがバレたんだろうか。

「なんとなく、訊いてみたくなったの。どう思う?」

「どうですかね、かっこいい部類には入らないと思いますが」

半分嘘で、半分本当だ。他の人がかっこいいと思わないことが、自分の恋心を否定する理由にはならない。

「そうかな。私はかっこいいと思うけどな、片岡くん」

目を逸らして隣にいる誰かに語りかけるようにそう言った。下手に茶化されている訳ではないらしい。どう返したものだろうか。

「物識りだし、面白い実験するし、好きなことに一直線で熱心に話しているところとか、私は好きだけどな」

「物識りは大槻先輩もそうじゃないですか」

「私はまだまだよ。片岡くんの方が知識に対して誠実だと思うわ」

「誠実……」

真意を図りかねている様子が伝わったのか、大槻先輩が添えるように言った。

「知識に対して誠実な人は、自分の知らないことに対して誠実な人ってことよ」

自分がそうではないと言っているにも拘らず、大槻先輩は少し嬉しそうに見えた。

「それは確かにすごいことだと思いますけど、でも私別に片岡先輩のことかっこいいとか、思ったりはしてないですから」

そう言うと片岡先輩に失礼な気もしたが、とりあえずその場をしのぎたかった。

「そう?そうなのね。じゃあ勘違いかもね。随分仲良くしてるみたいだから、少し気になっちゃって。今は二人きりだし、訊いてしまおうかなって」

何の意図もありませんと言ったような、涼しげな顔をしている。その態度に引っかかるものがあって、恵美はやり返してみることにした。

「先輩だって、村上先輩と仲がいいじゃないですか」

「村上先輩?そうね。でもそれは違うわ。私が彼を好きなんじゃなくて、彼が私のことを好きでいてくれるの」

「それって……」

声が詰まった。村上先輩が大槻先輩のことを好きなんだろうなって言うのは、見ていてよくわかる。でもそれを知ってて、自分は違うのに一方的に振り回していたっていうこと?

「村上先輩を弄ぶひどい人だと思う?」

心が読まれた。

「弄んでるとかそこまでは思いませんけど……」

咄嗟に嘘をついた。頭脳明晰、容姿端麗、性格柔和。その正体は悪魔だった。けれども、そんな人だったなんてまだ信じられない。

「いいのよ。わかってる、この話をするとみんなそう言うから。でも私はどうしていいかよくわからないの」

涼しげな顔が少し崩れた。大槻先輩が困っているところなんて、みたことがないかもしれない。

「私の好きな人にされたら嬉しいことをしているだけなの。話したり、わがまま言ったり、ちゃんとありがとうを言ったり。普通のことでしょ?そうしたら、私の好きな人にだって振り向いてもらえるかもしれない。でも、それがよくないのかもね」

「そんなの、逆じゃないですか。好きな人にされたら嬉しいことは、自分の好きな人にだけすればいいんです」

「そうもいかないわ。私のことが好きな男の人って、話したらわかっちゃうの。でも悪い人じゃないのに疎遠にするわけにもいかないでしょ。だって私が自分の好きな人から、好かれているっていう理由だけで嫌われたらいやだもの」

「それは、そうですけど……」

大槻先輩ほどモテると、いちいち振ってもいられないのだろうか。優しすぎて、相手からの気持ちを無下にできなくて、過剰に接してしまうだけ。でも人はそれを悪魔と言うだろう。

「やっぱり、先輩の好きな人にだけ特別扱いする方がいいと思います」

「そうね。でも相手から想われてないのかなって思うと、なんだか親しくもなれなくて。結局ずっと眺めているだけで終わっちゃうの」

「勿体無いですよ。先輩から好きって言われたら、どんな男だって」

「そんなことはないわ。わかるもの」

「わかるって、何がですか?」

「自分の好きな人が、誰のことを想っているかくらい、眺めていたらわかるわ」

整った顔に似合う憂鬱げなため息が聞こえた。

この人エスパーだろうか。私だって思ったことはある。人の気持ちがわかればどんなに楽だろうかと。でも実際のところ、目の前のエスパーはこんなに悩んでいる。

「そんなに簡単に、人の気持ちってわかるものなんですか」

「もちろん全ての人間の気持ちがわかるわけじゃないわ。私の好きな人と、私のことが好きな人のことだけ」

「それだけわかれば十分じゃないですか」

「いいえ、十分じゃないわ」

大槻先輩は一瞬目をつぶって、私の方をみた。

「例えば、あなたの好きな人が誰なのか、さっぱり皆目見当がつかないもの」

「そ、それは……。秘密、です」

「秘密なのね。また、もっと仲良くなったら教えてくれない?」

「それは、言えないかもしれません」

「でも、片岡くんじゃないのよね?」

「それは……」

大槻先輩が優しく問いかけてくる。でもそれは、返答を待っていると言うより、確認をしているかのようであった。

「言え、ません」

俯いて答えた。何もかも見透かしたような大槻先輩の目を直視することができない。

「なーんだ、そうなのね」

ふと顔をあげると、全てを納得したかのような表情でこちらを見ていた。

「じゃあ、今度また彼のこと、聞かせてね」

そう言うと、大槻先輩は立ち上がって荷物をまとめ始めた。

「もう定刻よ。待っていますから、そろそろ帰りましょう」

\vspace{0.2in}

帰り道、大槻先輩は今までなにも聞かなかったかのように、いつも通り他愛もない話をしてくれた。

紅茶に合うお菓子の話、成績の話、挙動のおかしいことで有名な先生の話。

自転車を押しながら、ひとしきり頷いて、ひとしきり笑って、そうしているうちに大槻先輩の乗るバス停まで来てしまった。

もうすぐバスがくるだろう。どうしても、確認しておきたい。

「あの、私からも一つ訊きたいことが」

ここで割り込まないと、もう一生このことを訊くチャンスは訪れない、そんな気がした。

「どうして、私の気持ちが知りたいんですか?」

「それはね」

大槻先輩はいつもの微笑みをたたえて一言返した。

「めぐみちゃんはもう、その理由を知っているんじゃないかしら」

バスが来てしまった。大槻先輩は、また明日、と言ってバスに乗って行った。


上の空でご飯を食べてお風呂に入ったら、観たいドラマも授業の予習や課題も放り出してベッドに飛び込んだ。

お母さんから「何をにやけているの」と不思議がられたけれど、そんなことも今はどうでもいい。

大槻先輩が私の気持ちを確認したい理由は、もう知っているはず。

そうなんだ。私の気持ちが知りたい理由なんて、一つしかない。

今日の会話を思い出しながら、大きなぬいぐるみの頭を抱いて、悶々とした気持ちを吐き出すようにアアアアと小さく叫んだ。

とっても嬉しい。にやけもするだろう。

明日は薬局に行ってにやけ止めクリームを買ってこないといけない。


ただ眺めているだけでよかったのに、私のことも見ていてくれたと言うことが嬉しくて。

明日から、どうやって接したらいいんだろう。いやいや、でもまだ確定したわけじゃないんだ。いつも通り、いつも通りだ。

眠たくてベッドに飛び込んだ訳ではないのだが、今夜は興奮して眠れそうにない。

放り出した予習をとりあえず終わらせて、眠れそうになってから寝よう。

そう決心してから一時間。まるで集中できない。結局何も手につかないまま、深夜の二時を迎えてとりあえず布団に入った。

今日の会話をなんども思い出しているうちに、いつのまにか眠りに落ちていた。翌朝、時計を見たら9時をゆうに過ぎていたが、土曜日だったので遅刻は免れた。


\end{document}